\documentclass[12pt]{article}
\usepackage[polish]{babel}
\usepackage[T1]{fontenc}
\usepackage[ top=30mm, bottom=30mm, right=20mm, left=20mm ]{geometry}
\usepackage{amsmath,amssymb}
\begin{document}

\section*{Całkowitoliczbowy model wyznaczania najtańszej \\ ścieżki z wymogiem ciągłego przedziału slice'ów}

\noindent \textbf{Zbiory i indeksy:}
\begin{align*}
	 & N: \text{zbiór węzłów (wierzchołków).}                      \\
	 & E \subseteq N \times N: \text{zbiór skierowanych krawędzi.} \\
	 & K = \{0,1,\dots,767\}: \text{zbiór indeksów slice'ów.}
\end{align*}

\vspace{1em}

\noindent \textbf{Parametry:}
\begin{align*}
	& w_{ij}
	& & \text{koszt (waga) użycia krawędzi }(i,j)\in E.\\
	&   & & \text{Wyliczanie wartości wagi rozwiązywane jest poza modelem całkowitoliczbowym} \\
	&   & & \text{i jest częścią wspólną logiki stosowanych optymalizatorów.} \\
	& s \in N
	& & \text{wierzchołek źródłowy (source).} \\
	& t \in N
	& & \text{wierzchołek docelowy (target).} \\
	& \text{O}_{(i,j),k} \in \{0,1\}
	& & \text{parametr binarny wskazujący, czy slice }k \text{ jest zajęty (}1\text{)} \\
	&   & & \text{czy wolny (}0\text{) na krawędzi }(i,j). \\
	& S \in \mathbb{Z}_{\ge 0}
	& & \text{liczba \emph{ciągłych} slice'ów wymagana na każdej wybranej krawędzi.}
\end{align*}

\vspace{1em}

\noindent \textbf{Zmienne decyzyjne:}
\begin{align*}
	 & x_{ij} \in \{0,1\}
	 &                         & \text{równe }1\text{, jeśli krawędź }(i,j)\text{ jest użyta w ścieżce; }0\text{ w p.p.}                                               \\
	 & y_{k} \in \{0,1\}
	 &                         & \text{równe }1\text{, jeśli slice }k \text{ jest wybrany jako \emph{początek} }                                                       \\
	 &                         &                                                                                         & \text{ciągłego przedziału; }0\text{ w p.p.} \\
	 & z_{(i,j),k} \in \{0,1\}
	 &                         & \text{zmienna pomocnicza do powiązania }x_{ij} \text{ i } y_k.
\end{align*}

\vspace{2em}

\noindent \textbf{Funkcja celu:}
\[
	\min \sum_{(i,j)\in E} w_{ij} \, x_{ij}.
\]
Jej celem jest zminimalizowanie łącznego kosztu (wagi) wszystkich wybranych krawędzi.

\vspace{2em}

\noindent \textbf{Ograniczenia:}

\begin{enumerate}
	\item \textbf{Zachowanie przepływu (równowaga przepływów).}

	      \[
		      \forall\, n \in N: \quad
		      \sum_{\substack{(n,j)\in E}} x_{n j}
		      \;-\;
		      \sum_{\substack{(i,n)\in E}} x_{i n}
		      =
		      \begin{cases}
			      1  & \text{jeśli } n = s,       \\
			      -1 & \text{jeśli } n = t,       \\
			      0  & \text{w przeciwnym razie}.
		      \end{cases}
	      \]
	      Zapewnia to, że ze źródła wypływa dokładnie 1 jednostka przepływu, do ujścia wpływa 1 jednostka, a pozostałe węzły mają bilans zerowy (co definiuje ścieżkę od \(\,s\) do \(\,t\)).

	      \vspace{1em}

	\item \textbf{Dostępność wolnych slice'ów na każdej wybranej krawędzi.}

	      Dla każdej krawędzi \((i,j)\in E\) musi istnieć co najmniej jeden indeks początkowy \(k\), dla którego (1) slice'y od \(k\) do \(k+S-1\) są wolne oraz (2) \(y_k=1\). W zapisie z uwzględnieniem iloczynu binarnych parametrów:
	      \[
		      \forall\,(i,j) \in E:\quad
		      \sum_{\substack{k \in K \\ k + S - 1 \,\le\, \max(K)}}
		      \Biggl(
		      y_k \cdot
		      \prod_{r = k}^{k + S - 1} \bigl[1 - \text{O}_{(i,j),r}\bigr]
		      \Biggr)
		      \;\;\ge\;\; x_{ij}.
	      \]
	      Jeśli \(x_{ij} = 1\), to co najmniej jeden blok \(S\) kolejnych wolnych slice'ów (począwszy od pewnego \(k\)) musi być dostępny.


	      \vspace{1em}

	\item \textbf{Zgodność wybranego slice'a z istniejącym zajęciem.}

	      \[
		      \forall\,(i,j)\in E,\;\forall\,k \in K: \;
		      \begin{aligned}
			       & \text{jeśli }(k + S - 1) > \max(K),\quad \text{pomijamy;}              \\
			       & \text{w przeciwnym razie } z_{(i,j),k} \;\le\; 1 - \text{O}_{(i,j),k}.
		      \end{aligned}
	      \]
	      Jeżeli slice \(k\) jest zajęty na krawędzi \((i,j)\), to zmienna \(z_{(i,j),k}\) musi wynosić 0.

	      \vspace{1em}

	\item \textbf{Powiązanie zmiennych \(z_{(i,j),k}\) z \(x_{ij}\) i \(y_k\).}

	      \[
		      \forall\,(i,j)\in E,\;\forall\,k \in K:\quad
		      \begin{cases}
			      z_{(i,j),k} \;\le\; y_k,    \\
			      z_{(i,j),k} \;\le\; x_{ij}, \\
			      z_{(i,j),k} \;\ge\; y_k + x_{ij} - 1.
		      \end{cases}
	      \]
	      Jest to standardowe odwzorowanie równości
	      \[
		      z_{(i,j),k} = x_{ij} \cdot y_k.
	      \]

	      \vspace{1em}

	\item \textbf{Wzajemna wyłączność wybranego slice'a początkowego.}

	      \[
		      \sum_{\substack{k \in K \\ k + S - 1 \,\le\, \max(K)}}
		      y_k
		      \;=\; 1.
	      \]
	      Dokładnie jeden slice \(k\) zostaje wybrany jako początek bloku długości \(S\).

\end{enumerate}

\end{document}
